\documentclass[journal,12pt,twocolumn]{IEEEtran}
\usepackage{setspace}
\usepackage{gensymb}
\usepackage{caption}
\singlespacing
\usepackage{csvsimple}
\usepackage{amsmath}
\usepackage{multicol}
\usepackage{amssymb}
\usepackage{newfloat}
\usepackage{listings}
%\usepackage{iithtlc}
\usepackage{color}
\usepackage[colorlinks= true,
			linkcolor = blue,
			urlcolor = blue,
			citecolor = blue
			anchorcolor = blue]{hyperref}
\usepackage{tikz}
\usetikzlibrary{shapes,arrows}
\usepackage{hyperref}
\usepackage{amsthm}
\usepackage{mathrsfs}
\usepackage{txfonts}
\usepackage{stfloats}
\usepackage{cite}
\usepackage{cases}
\usepackage{mathtools}
\usepackage{caption}
\usepackage{enumerate}	
\usepackage{enumitem}
\usepackage{amsmath}

\usepackage{longtable}
\usepackage{multirow}

\usepackage{enumitem}
\usepackage{mathtools}
\usepackage{hyperref}
%\usepackage[framemethod=tikz]{mdframed}
\usepackage{listings}
    %\usepackage[latin1]{inputenc}                                 %%
    \usepackage{color}                                            %%
    \usepackage{array}                                            %%
    \usepackage{longtable}                                        %%
    \usepackage{calc}                                             %%
    \usepackage{multirow}                                         %%
    \usepackage{hhline}                                           %%
    \usepackage{ifthen}                                           %%
  %optionally (for landscape tables embedded in another document): %%
    \usepackage{lscape}     


\usepackage{url}
\def\UrlBreaks{\do\/\do-}


%\usepackage{stmaryrd}


%\usepackage{wasysym}
%\newcounter{MYtempeqncnt}
\DeclareMathOperator*{\Res}{Res}
%\renewcommand{\baselinestretch}{2}
\renewcommand\thesection{\arabic{section}}
\renewcommand\thesubsection{\thesection.\arabic{subsection}}
\renewcommand\thesubsubsection{\thesubsection.\arabic{subsubsection}}

\renewcommand\thesectiondis{\arabic{section}}
\renewcommand\thesubsectiondis{\thesectiondis.\arabic{subsection}}
\renewcommand\thesubsubsectiondis{\thesubsectiondis.\arabic{subsubsection}}

% correct bad hyphenation here
\hyphenation{op-tical net-works semi-conduc-tor}

%\lstset{
%language=C,
%frame=single, 
%breaklines=true
%}

%\lstset{
	%%basicstyle=\small\ttfamily\bfseries,
	%%numberstyle=\small\ttfamily,
	%language=Octave,
	%backgroundcolor=\color{white},
	%%frame=single,
	%%keywordstyle=\bfseries,
	%%breaklines=true,
	%%showstringspaces=false,
	%%xleftmargin=-10mm,
	%%aboveskip=-1mm,
	%%belowskip=0mm
%}

%\surroundwithmdframed[width=\columnwidth]{lstlisting}
\def\inputGnumericTable{}                                 %%
\lstset{
%language=C,
frame=single, 
breaklines=true,
columns=fullflexible
}
 

\begin{document}
%
\tikzstyle{block} = [rectangle, draw,
    text width=3em, text centered, minimum height=3em]
\tikzstyle{sum} = [draw, circle, node distance=3cm]
\tikzstyle{input} = [coordinate]
\tikzstyle{output} = [coordinate]
\tikzstyle{pinstyle} = [pin edge={to-,thin,black}]

\theoremstyle{definition}
\newtheorem{theorem}{Theorem}[section]
\newtheorem{problem}{Problem}
\newtheorem{proposition}{Proposition}[section]
\newtheorem{lemma}{Lemma}[section]
\newtheorem{corollary}[theorem]{Corollary}
\newtheorem{example}{Example}[section]
\newtheorem{definition}{Definition}[section]
%\newtheorem{algorithm}{Algorithm}[section]
%\newtheorem{cor}{Corollary}
\newcommand{\BEQA}{\begin{eqnarray}}
\newcommand{\EEQA}{\end{eqnarray}}
\newcommand{\define}{\stackrel{\triangle}{=}}
\bibliographystyle{IEEEtran}
%\bibliographystyle{ieeetr}
\providecommand{\nCr}[2]{\,^{#1}C_{#2}} % nCr
\providecommand{\nPr}[2]{\,^{#1}P_{#2}} % nPr
\providecommand{\mbf}{\mathbf}
\providecommand{\pr}[1]{\ensuremath{\Pr\left(#1\right)}}
\providecommand{\qfunc}[1]{\ensuremath{Q\left(#1\right)}}
\providecommand{\sbrak}[1]{\ensuremath{{}\left[#1\right]}}
\providecommand{\lsbrak}[1]{\ensuremath{{}\left[#1\right.}}
\providecommand{\rsbrak}[1]{\ensuremath{{}\left.#1\right]}}
\providecommand{\brak}[1]{\ensuremath{\left(#1\right)}}
\providecommand{\lbrak}[1]{\ensuremath{\left(#1\right.}}
\providecommand{\rbrak}[1]{\ensuremath{\left.#1\right)}}
\providecommand{\cbrak}[1]{\ensuremath{\left\{#1\right\}}}
\providecommand{\lcbrak}[1]{\ensuremath{\left\{#1\right.}}
\providecommand{\rcbrak}[1]{\ensuremath{\left.#1\right\}}}
\theoremstyle{remark}
\newtheorem{rem}{Remark}
\newcommand{\sgn}{\mathop{\mathrm{sgn}}}
\providecommand{\abs}[1]{\left\vert#1\right\vert}
\providecommand{\res}[1]{\Res\displaylimits_{#1}} 
\providecommand{\norm}[1]{\left\Vert#1\right\Vert}
\providecommand{\mtx}[1]{\mathbf{#1}}
\providecommand{\mean}[1]{E\left[ #1 \right]}
\providecommand{\fourier}{\overset{\mathcal{F}}{ \rightleftharpoons}}
%\providecommand{\hilbert}{\overset{\mathcal{H}}{ \rightleftharpoons}}
\providecommand{\system}{\overset{\mathcal{H}}{ \longleftrightarrow}}
	%\newcommand{\solution}[2]{\textbf{Solution:}{#1}}
\newcommand{\solution}{\noindent \textbf{Solution: }}
\newcommand{\myvec}[1]{\ensuremath{\begin{pmatrix}#1\end{pmatrix}}}
\providecommand{\dec}[2]{\ensuremath{\overset{#1}{\underset{#2}{\gtrless}}}}
\DeclarePairedDelimiter{\ceil}{\lceil}{\rceil}
%\numberwithin{equation}{section}
%\numberwithin{problem}{subsection}
%\numberwithin{definition}{subsection}
\makeatletter
\@addtoreset{figure}{section}
\makeatother
\let\StandardTheFigure\thefigure
\renewcommand{\thefigure}{\thesection}
\numberwithin{problem}{section}
\makeatletter
\@addtoreset{table}{section}
\makeatother
\let\StandardTheFigure\thefigure
\let\StandardTheTable\thetable
\let\vec\mathbf
\numberwithin{equation}{section}
\vspace{3cm}
\title{
	\logo{
	Random Numbers
	}
}

\author{Aniket Tukaram Satpute}

%\maketitle
\begin{abstract}
This manual provides a simple introduction to the generation of random numbers
\end{abstract}
%%
\section{Uniform Random Numbers}
Let $U$ be a uniform random variable between 0 and 1.
\begin{enumerate}[label=\thesection.\arabic*
,ref=\thesection.\theenumi]
%1.1
\item Generate $10^6$ samples of $U$ using a C program and save into a file called uni.dat .
\\
\solution : Link to the code : \href{https://github.com/anikettsatpute/Probability-and-Random-Variable-Assignment/blob/main/code/code1_1.c}{C code}
%
\vspace{0.2in}
%1.2
\item
Load the uni.dat file into python and plot the empirical CDF of $U$ using the samples in uni.dat. The CDF is defined as
\begin{align}
F_{U}(x) = \pr{U \le x}
\end{align}
\\
\solution  The following code plots Fig: \href{https://github.com/anikettsatpute/Probability-and-Random-Variable-Assignment/blob/main/code/code1_2.py}{Python code} 
\begin{figure}[h]
\includegraphics[width=\columnwidth]{../Figures/uni_cdf}
\caption{The CDF of $U$}
\label{fig:uni_cdf}
\end{figure}
%
%1.3
\item
Find a  theoretical expression for $F_{U}(x)$.\\
\solution As U is uniformly distributed random variable in the interval (0,1) and
\begin{align}
p_U(x) = 
\begin{cases}
1, & x \in (0, 1) \\
0, & otherwise
\end{cases}
\end{align}
Hence,
\begin{align}
F_U(x) = 
\begin{cases}
0, & x \in (-\infty,0) \\
x, & x \in (0,1)\\
1, & x \in (1,\infty)
\end{cases}
\end{align}

%1.4
\item
The mean of $U$ is defined as
\begin{equation}
E\sbrak{U} = \frac{1}{N}\sum_{i=1}^{N}U_i
\end{equation}
and its variance as
\begin{equation}
\text{var}\sbrak{U} = E\sbrak{U- E\sbrak{U}}^2 
\end{equation}
Write a C program to  find the mean and variance of $U$.\\

\solution Link to the code :\href{https://github.com/anikettsatpute/Probability-and-Random-Variable-Assignment/blob/main/code/code1_4.c}{C code}

\begin{figure}[h]
\centering
\includegraphics[width=\columnwidth]{../fig/screen15.png}
\caption{Output}
\label{fig:gauss_cdf}
\end{figure}

%1.5
\item Verify your result theoretically given that
\end{enumerate}

\begin{equation}
E\sbrak{U^k} = \int_{-\infty}^{\infty}x^kdF_{U}(x)
\end{equation}

\solution we know that,
\begin{equation}
dF_{U}(x) = p_U(x)dx
\end{equation}
also mean ($\mu$) is E(U):
Hence,
\begin{align*}
\mu &= \int_{-\infty}^{\infty}xp_{U}(x)dx\\
&= \int_{0}^{1}xdx\\
&= \frac{x^2}{2} \Bigg|^{1}_{0} \\
&= \frac{1}{2}
\end{align*}
Variance($\sigma^{2}$) = $E(U^{2}) - E(U)^{2}$
\begin{align*}
E(U^{2}) &= \int_{-\infty}^{\infty}x^{2}p_{U}(x)dx\\
&= \int_{0}^{1}x^{2}dx\\
&= \frac{x^3}{3} \Bigg|^{1}_{0} \\
&= \frac{1}{3}
\end{align*}
\begin{align*}
\sigma^{2} &= \frac{1}{3} - \frac{1}{4}\\
&= \frac{1}{12}
\end{align*}
\vspace{0.3in}
\section{Central Limit Theorem}
%
\begin{enumerate}[label=\thesection.\arabic*
,ref=\thesection.\theenumi]



%2.1
\item
Generate $10^6$ samples of the random variable
\begin{equation}
X = \sum_{i=1}^{12}U_i -6
\end{equation}
using a C program, where $U_i, i = 1,2,\dots, 12$ are  a set of independent uniform random variables between 0 and 1
and save in a file called gau.dat\\
\solution  link to the code :\href{https://github.com/anikettsatpute/Probability-and-Random-Variable-Assignment/blob/main/code/code2_1.c}{C code}
\vspace{0.2in}

%2.2
\item
Load gau.dat in python and plot the empirical CDF of $X$ using the samples in gau.dat. What properties does a CDF have?
\\
\solution Link to Python code :\href{https://github.com/anikettsatpute/Probability-and-Random-Variable-Assignment/blob/main/code/code2_2.py}{Python code}

\begin{figure}[h]
\centering
\includegraphics[width=\columnwidth]{../Figures/gau_cdf}
\caption{The CDF of $X$}
\label{fig:gau_cdf}
\end{figure}

Properties:\\
(1)  Graph is symmetric about a single point\\
(2)	 The $F_X(x)$ is non-decreasing function\\
(3)	 $\lim_{x \to \infty} F_X(x)$
\vspace{0.2in}

%2.3
\item
Load gau.dat in python and plot the empirical PDF of $X$ using the samples in gau.dat. The PDF of $X$ is defined as
\begin{align}
p_{X}(x) = \frac{d}{dx}F_{X}(x)
\end{align}
What properties does the PDF have?
\\
\solution Link to the code :\href{https://github.com/anikettsatpute/Probability-and-Random-Variable-Assignment/blob/main/code/code2_3.py}{Python code}

\begin{figure}[h]
\centering
\includegraphics[width=\columnwidth]{../Figures/gauss_pdf}
\caption{The PDF of $X$}
\label{fig:gauss_pdf}
\end{figure}

Properties :\\
(1) Area under the curve is One.\\
(2) Symmetric about line $x=\mu$.\\
(3) Increasing in first half and decreasing in other half.

\vspace{0.2in}

%2.4
\item Find the mean and variance of $X$ by writing a C program.\\
\solution Link to the code :\href{https://github.com/anikettsatpute/Probability-and-Random-Variable-Assignment/blob/main/code/code2_4.c}{C code}
\begin{figure}[h]
\centering
\includegraphics[width=\columnwidth]{../fig/screen25.png}
\caption{Output}
\label{fig:gauss_cdf}
\end{figure}

%2.5
\item Given that
\begin{align}
p_{X}(x) = \frac{1}{\sqrt{2\pi}}\exp\brak{-\frac{x^2}{2}}, -\infty < x < \infty,
\end{align}
repeat the above exercise theoretically.
\solution \\
(1) CDF is 
\begin{align*}
F_X(x) &= \int_{-\infty}^{\infty}p_X(x)dx\\
&= 1
\end{align*}
(2) Mean is
\begin{align*}
\mu = E(x) = \int_{-\infty}^{\infty} x p_X(x)dx\\
\end{align*}
Due to symmetry clearly, $\mu = 0$\\

(3) Variance is
\begin{align*}
var[X] &= E(X^{2}) - E(X)^{2}\\
&= \int_{-\infty}^{\infty} x^{2} p_X(x)dx  - 0\\
&= 1
\end{align*}


\end{enumerate}
\section{From Uniform to Other}
\begin{enumerate}[label=\thesection.\arabic*
,ref=\thesection.\theenumi]

%3.1
\item
Generate samples of 

\begin{equation}
V = -2\ln\brak{1-U}
\end{equation}

and plot its CDF. \\
\solution Link to the code :\href{https://github.com/anikettsatpute/Probability-and-Random-Variable-Assignment/blob/main/code/code3_2.c}{Python code}
\begin{figure}[h]
\centering
\includegraphics[width=\columnwidth]{../Figures/3_1_cdf}
\caption{CDF of V}
\label{fig:gauss_pdf}
\end{figure}

%3.2
\item Find a theoretical expression for $F_V(x)$.\\
\solution \\
\begin{align*}
F_V(X) &= Pr(V \le x)\\
&= Pr(-2ln(1-U) \le x)\\
&= Pr((1-U) \ge exp(\frac{-x}{2})\\
&= Pr(U \le (1-exp(\frac{-x}{2}))\\
&= F_U(1-exp \frac{-x}{2})
\end{align*} 
from equation (1.3),
\begin{align*}
F_U(x) = 
\begin{cases}
0, & x \in (-\infty,0) \\
(1-exp \frac{-x}{2}), & x \in (0,\infty)
\end{cases}
\end{align*}
\end{enumerate}
%\section{Triangular Distribution}
%\begin{enumerate}

%4.1
%\item Generate
%\begin{equation}
%T = U_1 + U_2
%\end{equation}
%\solution Link to the code :\href{https://github.com/anikettsatpute/Probability-and-Random-Variable-Assignment/blob/main/code/code4_1.c}{C code}
%\vspace{0.2in}
%4.2
%\item Find CDF of T\\
%\solution Link to the code :\href{https://github.com/anikettsatpute/Probability-and-Random-Variable-Assignment/blob/main/code/code4_2.py}{Pyhton code}\\

%\begin{figure}[h]
%\centering
%\includegraphics[width=\columnwidth]{../Figures/4_3_CDF}
%\caption{CDF of T}
%\label{fig:gauss_pdf}
%\end{figure}



%\end{enumerate}
\end{document}
